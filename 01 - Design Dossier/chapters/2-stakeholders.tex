\begin{fullwidth}
\section{Stakeholders}
\subsection{Point of Contact}
    The Point-of-Contact (POC) is Wilford Erasmus, Chief of Operations from the U.S. Department of Homeland Security and the U.S. Customs and Border Protection. The Chief of Operations is representing a 3-letter agency and is accountable for the project and customer interface. 

\subsection{The Customer}
    The Customer is a 3-letter agency represented by the U.S. Department of Homeland Security.

\subsection{Other Stakeholders}
    An investigation of other stakeholders yielded individuals and groups that Futuristic Innovative Technologies (FIT) may need to consult with, have negative interest in the project succeeding, or may benefit from the success of the project. Table \ref{tab:other_stakeholders} summarizes the other stakeholders for this project.
    \begin{longtable}{ | p{5cm} | p{12cm} | }
        \hline
        \rowcolor[gray]{.8}
        \textbf{Stakeholder Class} &  \textbf{Stakeholder Description} \\
        \hline
        External consultant & 
        \begin{itemize}
            \item Commercial Off The Shelf (COTS) suppliers
            \item Technical Subject Matter Experts (SME); if necessary
            \item Federal Aviation Administration (FAA)
            \item Federal Communications Commission (FCC)
            \item General Atomics - original developer of the AUAV
            \item National Weather Service (NWS)
            \item United States Embassies
            \item National Society of Professional Surveyors
            \item Air Force Research Laboratory (AFRL)
            \item Air Force Life Cycle Management Center (AFLCMC)
            \item Department of Transportation
        \end{itemize} \\
        \hline
        Negative stakeholders &
        \begin{itemize}
            \item Competitors (Northrup Grumman and other bidders)
            \item Civil liberty groups (e.g. American Civil Liberty Union)
            \item Citizen privacy groups (e.g. Electronic Privacy Information Center)
            \item Political parties (e.g. Democratic Party of America)
            \item Environmental and Ecological Organizations
        \end{itemize} \\
        \hline
        Functional Beneficiaries &
        Primary Stakeholders:
        \begin{itemize}
            \item 3-letter agencies
            \item International universities
            \item International research organizations
            \item International air and space agencies
            \item United States' allies and partners
            \item United States military
        \end{itemize} \\
         &
        Secondary Stakeholders:
        \begin{itemize}
            \item National Security Agencies (e.g. HSBP, FBI, CIA, ATF, DEA, DIA, US Marshals Service, local law enforcement, etc.)
            \item First responders (United States and international)
            \item United States Coast Guard
            \item National Park Service
            \item Aid organizations (e.g. USAID, Doctors Without Borders, etc.)
            \item Interpol
        \end{itemize} \\
        \hline
        \caption{Table of Other Stakeholders}
        \label{tab:other_stakeholders}
    \end{longtable}
    
    It is expected that the Federal Aviation Administration (FAA) will need to be consulted to obtain airframe and airworthiness certifications and that the State and Municipal Governments of the US. The Federal Communications Commission (FCC) will be consulted to obtain transmitting licenses. As well, FIT will need to consult with COTS suppliers for system components. Depending on the requirements, FIT may need to contract with, or sub-contract work to, subject matter experts (SMEs) to resolve deficiencies during the project.  General Atomics as the designer of the UAV being modified will need to be consulted for drawings, cable routing and other issues that will impact the design. Applicable Embassies will also need to be contacted to ensure proper permissions and certifications for overseas deployment.
    
    Although we believe that all design work can be performed in-house, FIT should be aware that Northrop Grumman and any other bidders are negative stakeholders that FIT may need to contract out for specialized components to adhere to schedule, cost, quality, and risk expectations are met, although direct competitors in the AUAV sector. The other negative stakeholders and functional beneficiaries will not be actively engaged in the project otherwise.
    
    Especially considering the benign appearance of the AUAV and the advanced search and rescue functions, the product may be of interest to agencies and organizations that regularly participate in rescue efforts in extreme environments. The product may be beneficial to these stakeholders with minimal changes or updates. 
    
    \subsection{The Hands-On Users}
    The project has identified five (5) entities that are the hands-on users. These are AUAV Flight Officers, AUAV Optical Operator, Intelligence Officers, AUAV Maintenance Technicians, and Mission Technicians

    Table \ref{tab:hands_on_users} shows each of the hands-on users’ current roles and responsibilities
    
    \begin{longtable}{ | p{5cm} | p{12cm} | }
        \hline
        \rowcolor[gray]{.8}
        \textbf{Stakeholder} & \textbf{Current Tasks} \\
        \hline
        AUAV Flight Officer &
        \begin{itemize}
            \item Remotely pilot the AUAV through taxiing, take-off, in-fight operations, and landing when the option for manual control is selected.
            \item Direct the AUAV to locations as ordered
            \item Update mission flight plans as ordered
            \item Select visibility mode (e.g. electro-optical during the day, Infrared radiation (IR) at night, or a combination at both in low-light times (e.g. dusk/dawn)
        \end{itemize} \\
        \hline
        AUAV Optical Operator &
        \begin{itemize}
            \item Orient electro-optical imaging on AUAV to view targets of interest
            \item Optimize and modify settings of AUAV during flight operations
            \item Monitor AUAV data streams and notify Flight Officer of situational anomalies
            \item Monitor AUAV status and direct to maintenance, as necessary
            \item Lock onto and track a target  while it remains stationary or move, relative to the AUAV
        \end{itemize} \\
        \hline
        Intelligence Officer &
        \begin{itemize}
            \item Interpret images from AUAV to confirm bipedal movement
            \item Initiate contact with land-based Mobile Command Centers (MCC)
            \item Confirm AI identification and detection of target
            \item Use obtained intelligence to guide AUAV operations and recommend course of action
            \item Verify operations are proceeding as planned
            \item Improvise new operational plan as situation and superiors dictate
        \end{itemize} \\
        \hline
        AUAV Maintenance Technicians &
        \begin{itemize}
            \item Perform pre and post-flight checkouts
            \item Fill fuel tanks, oil reservoirs, and other consumables
            \item Start and stop the AUAV engine
            \item Perform preventative maintenance, diagnostics, repairs, and servicing
            \item Help troubleshoot any mechanical failure during an operation
        \end{itemize} \\
        \hline
        Mission Technicians &
        \begin{itemize}
            \item Perform preventative maintenance, diagnostics, and repairs to the AUAV MCC, communication links, and associated systems
        \end{itemize} \\
        \hline
        \caption{Hands-On Users Current Roles and Responsibilities}
        \label{tab:hands_on_users}
    \end{longtable}
\end{fullwidth}